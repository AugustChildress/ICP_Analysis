To determine the distance to the supernova, we need to know a 
few important things for each day/data point of observation:
\\
We need:
\\
\\
$v$: The velocity of the expanding gas, Units: km/s.
\\
$t$: The time since explosion, Units: Julian Date.
\\
$t_0$: The time of explosion, Units: Julian Date.
\\
$z$: The time dialation factor, Unitless.
\\
$\theta$: The angular radius, Units: Degrees.
\\
\\
As seen in the figure \ref{fig:distance_graph}, we have 5 data points. 
For each of these data points, we have a $t$, $v$, and $\theta$.
\\
\\
First we start with velocity $v$. This is determined by looking at 
the spectrum of the expanding photosphere and seeing how much it is blueshifted compared to 
a source nearby. This will tell us how fast the expanding gas is moving towards us, thus 
giving us a velocity.
\\
\\
Next we will find the time $t$, which is as easy as using a device that tells you the 
julian date\footnote{Julian date is the number of days (and fractions of days) 
since 12:00 PM November 24, 4714 BC.} of the time of observation. 
\\
\\
$t_0$ is a bit harder to find because we dont always see exactly when a supernova 
happens, often we find out a few dats after the fact. So for this analysis we used two methods.
\\
Method 1: 
\\
We calculate the line of best fit normally, which tells us the distance $d$ and the time of explosion $t_0$.
\\
Method 2: 
We calculate the line of best fit but set the y-intercept, $t_0$, to zero, and calculated the distance.
\\
\\
The time dialation factor $z$ is calculated using spectroscopy, looking at spectrums of the light and determing 
how much time will be dialated from traveling from the direction of the supernova. This is something that needs 
to be looked up. We did not calculate this ourselves.
\\
\\
Now, we have the angular radius $\theta$. Determinig the angular radius is the most difficult part of this project,
and requires a large amount of data. To start off, we will look at the equation for calculating $\theta$